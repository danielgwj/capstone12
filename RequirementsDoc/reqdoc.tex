\documentclass[10pt,journal,compsoc]{IEEEtran}

% *** CITATION PACKAGES ***
\ifCLASSOPTIONcompsoc
  % IEEE Computer Society needs nocompress option
  % requires cite.sty v4.0 or later (November 2003)
  \usepackage[nocompress]{cite}
\else
  % normal IEEE
  \usepackage{cite}
\fi

\usepackage{textcomp}
\usepackage{geometry}
\geometry{margin=0.75in}

\begin{document}
\onecolumn

\begin{titlepage}

\begin{flushright}
\textbf{IEEE Std 830-1998} \\
(Revision of \\
IEEE Std 830-1993) \\
\vspace{5mm}
\textbf{IEEE Std 830-1998}
\end{flushright}

\vspace{25mm}

\begin{flushleft}
\begin{bfseries}
	\vskip2mm
	\Huge{Requirements Document for\\ Better Graphics For A Robotics Grasping GUI}\\
	\vspace{30mm}
	\textbf{\huge Shady Robots} \\
	
\end{bfseries}

\vspace{15mm}
\Large{CS461: Senior Software Engineering Project} \\
\Large{Fall 2016} \\

\vspace{10mm}

\today

\end{flushleft}

\newpage

\begin{flushright}
\textbf{IEEE Std 830\texttrademark-1998(R2009)} \\
(Revision of \\
IEEE Std 830-1993) \\
\end{flushright}

\vspace{15mm}

\begin{flushleft}
\begin{bfseries}
	\vskip2mm
	\Huge{Requirements Document for\\ Better Graphics For A Robotics Grasping GUI}\\
	\vspace{30mm}
	\textbf{\huge Shady Robots} \\
	
\end{bfseries}

\vspace{15mm}
\Large{CS461: Senior Software Engineering Project} \\
\Large{Fall 2016} \\

\vspace{10mm}

\today

\vfill

\begin{large}
{\bf Abstract:}
Humans are good at interacting with objects, however they aren't good at explaining the thought behind the action.
Robots are bad at grasping objects.
Thus, a simulation has been created to help improve robot grasping algorithms by collecting knowledge about the human decision making process.
However, the simulation (OpenRave) currently has outdated graphics.
Our plan is to take this existing robot grasping visualization interface and augment it with warm cool shaders, shadows and silhouettes.
Warm cool shaders add contrast to objects while still maintaining their visibility.
Shadows highlight the position and movement of objects in the simulation.
Silhouettes make objects visually distinct from one another.
All of these additions will help users understand the geometric relationship between the grasping hand and the object it is trying to pick up. 
Our goal is to increase user confidence when answering questions about the simulation.

{\bf Keywords:} OpenGL, OpenRave, shaders, silhouettes, shadows, robotic simulation,
software requirements specification, system requirements specifications
\end{large}
\end{flushleft}

\newpage


\end{titlepage}

\section*{Introduction}


\section*{Participants}

\newpage

\tableofcontents


\newpage

\section{Overview}

\vfill

\section{Definitions}

\vfill

\newpage

\section{Specific requirements}

\subsection{External interface requirements}
\subsubsection{User interfaces}
\subsubsection{Hardware interfaces}
\subsubsection{Software interfaces}
\subsubsection{Communications interfaces}

\vfill

\subsection{Functional requirements}

\vfill

\subsection{Performance requirements}

\vfill

\subsection{Design constraints}

\vfill

\subsection{Software system attributes}

\vfill

\subsection{Other requirements}

\vfill

\newpage

\noindent\begin{tabular}{ll}
\makebox[2.5in]{\hrulefill} & \makebox[2.5in]{\hrulefill}\\
Cindy Grimm & Date\\[4ex]% adds space between the signatures
\makebox[2.5in]{\hrulefill} & \makebox[2.5in]{\hrulefill}\\
Justin Bibler & Date\\[4ex]% adds space between the signatures
\makebox[2.5in]{\hrulefill} & \makebox[2.5in]{\hrulefill}\\
Matthew Huang & Date\\[4ex]% adds space between the signatures
\makebox[2.5in]{\hrulefill} & \makebox[2.5in]{\hrulefill}\\
Daniel Goh & Date\\
\end{tabular}

\end{document}



