\documentclass[10pt,journal,compsoc]{IEEEtran}

% *** CITATION PACKAGES ***
\ifCLASSOPTIONcompsoc
  % IEEE Computer Society needs nocompress option
  % requires cite.sty v4.0 or later (November 2003)
  \usepackage[nocompress]{cite}
\else
  % normal IEEE
  \usepackage{cite}
\fi

\usepackage{lscape}
\usepackage{textcomp}
\usepackage{pgfgantt}
\usepackage{geometry}
\geometry{margin=0.75in}

\begin{document}
\onecolumn

\begin{titlepage}

\begin{flushright}
\textbf{IEEE Std 830-1998} \\
(Revision of \\
IEEE Std 830-1993) \\
\vspace{5mm}
\textbf{IEEE Std 830-1998}
\end{flushright}

\vspace{25mm}

\begin{flushleft}
\begin{bfseries}
	\vskip2mm
	\Huge{Requirements Document for\\ Better Graphics For A Robotics Grasping GUI}\\
	\vspace{30mm}
	\textbf{\huge Shady Robots} \\
	
\end{bfseries}

\vspace{15mm}
\Large{CS461: Senior Software Engineering Project} \\
\Large{Fall 2016} \\

\vspace{10mm}

\today

\end{flushleft}

\newpage

\begin{flushright}
\textbf{IEEE Std 830\texttrademark-1998(R2009)} \\
(Revision of \\
IEEE Std 830-1993) \\
\end{flushright}

\vspace{15mm}

\begin{flushleft}
\begin{bfseries}
	\vskip2mm
	\Huge{Requirements Document for\\ Better Graphics For A Robotics Grasping GUI}\\
	\vspace{30mm}
	\textbf{\huge Shady Robots} \\
	
\end{bfseries}

\vspace{15mm}
\Large{CS461: Senior Software Engineering Project} \\
\Large{Fall 2016} \\

\vspace{10mm}

\today

\vfill

\begin{normalsize}
{\bf Abstract:}
Humans are good at interacting with objects, however they aren't good at explaining the thought behind the action.
Robots are bad at grasping objects.
Thus, a simulation has been created to help improve robot grasping algorithms by collecting knowledge about the human decision making process.
However, the simulation (OpenRave) currently has outdated graphics.
Our plan is to take this existing robot grasping visualization interface and augment it with warm cool shaders, shadows and silhouettes.
Warm cool shaders add contrast to objects while still maintaining their visibility.
Shadows highlight the position and movement of objects in the simulation.
Silhouettes make objects visually distinct from one another.
All of these additions will help users understand the geometric relationship between the grasping hand and the object it is trying to pick up. 
Our goal is to increase user confidence when answering questions about the simulation.

{\bf Keywords:} OpenInventor, OpenGL, OpenRave, shaders, warm cool shaders, silhouettes, shadows, robotic simulation, geometry, visualization, render,
software requirements specification, system requirements specifications
\end{normalsize}
\end{flushleft}

\newpage

\end{titlepage}

\section*{Introduction}
\vspace{3mm}
Currently, our client is using visualizations, of a robot hand grasping objects, to collect data online.
This data is used to create a model of the human grasp.
However, the current graphics in the visualizations are outdated.
It is hard to see and understand the shapes and contact points represented in the scene.
Because of this, users aren't confident in giving proper responses to the simulation.
Thus, the graphics need to be updated to allow meaningful data to be collected. \\

\section*{Participants}
\vspace{3mm}
Cindy Grimm, Matthew Huang, Daniel Goh and Justin Bibler \\ 


\newpage

\tableofcontents

\newpage

\section{Overview}
\vspace{3mm}
This document describes requirements needed to enhance the current openrave simulation. 
Clause 2 lists the references made to other documents. 
Clause 3 provides definitions of specific terms used.
Clause 4 lists the specific requirements that will be met.
This includes the external interface requirements, functional requirements, performance requirements, design constraints, software system attributes, and other requirements.

\vfill

\section{References}

\vfill

\section{Definitions}
\begin{flushleft}
\vspace{3mm}
\textbf{Contract:}
A legally binding document agreed upon by the customer and supplier. This includes the technical and organizational requirements, cost, and schedule for a product. A contract may also contain informal but useful information such as the commitments or expectations of the parties involved.

\vspace{3mm}
\textbf{Customer:}
The person, or persons, who pay for the product and usually (but not necessarily) decide the requirements. In the context of this recommended practice the customer and the supplier may be members of the same organization.

\vspace{3mm}
\textbf{Supplier:}
The person, or persons, who produce a product for a customer. In the context of this recommended practice, the customer and the supplier may be members of the same organization.

\vspace{3mm}
\textbf{User:}
The person, or persons, who operate or interact directly with the product. The user(s) and the customer(s) are often not the same person(s).

\end{flushleft}
\vfill

\newpage

\section{Specific requirements}
\vspace{3mm}

\subsection{External interface requirements}
\vspace{3mm}
The purpose of the software's output is to give a visual understanding of the robot's grasping capabilities.
The software will accept an input simulation file (.py file) through the console.
The software will then output a simulation onto a new window.

\vfill

\subsection{Functional requirements}
\vspace{3mm}
The system shall render the scene and output to a new window.
The rendered scene shall include all 3D objects to be displayed in the scene with correct* geoemtry.
The scene shall also include:
\begin{itemize}
\item Warm cool shaders
\item Shadows
\item Silhouettes
\end{itemize}
The system shall continously rerender the scene until completion of the simulation or software termination.

\vfill

\subsection{Performance requirements}
\vspace{3mm}
The system output should maintain a minimum stable 30FPS on an average modern computer*.
The system will only run one simulation per window.
The initial scene will be rendered within the first 30 seconds after accepting an input file.

\vfill

\subsection{Design constraints}
\vspace{3mm}
The system currently utilizes outdated OpenGL libraries.
The location of the main render loop is not known to the customer.

\vfill

\subsection{Software system attributes}
\vspace{3mm}
\subsubsection{Reliability}
\vspace{5mm}
The system shall always run properly created simulations.

\subsubsection{Maintainability}
\vspace{3mm}
The implemented system will be easily maintained and upgraded. 
Functions should be easily testable and well documented.
Functions should have single, clear purposes; they should not overlap in what they do.


\subsubsection{Portability}
\vspace{3mm}
The system shall work on modern Windows operating systems as well as Linux.

\vfill

\subsection{Other requirements}
\vspace{3mm}
The outdated OpenGL libraries will be updated to utilize more modern libraries.

\vfill

\newpage

\begin{landscape}
\section{Gantt Chart}
\begin{ganttchart}{27}
\gantttitle{2016}{9} 
\gantttitle{2017}{18} \\
\gantttitlelist{10,...,12}{3} 
\gantttitlelist{1,...,6}{3} \\
\ganttgroup{Shady Robots}{1}{27} \\
\ganttbar{Explore and familiarize knowledge with the simulation}{1}{6} \\ % end of november
\ganttlinkedbar{Implement warm cool shaders in simulation}{7}{12} \\ % 2 month
\ganttmilestone{Warm Cool Shaders Implementation Milestone}{12} \\
\ganttlinkedbar{Implement silhouettes in simulation}{13}{18} \\ % 1 month
\ganttmilestone{Silhouettes Implementation Milestone}{18} \\
\ganttbar{Implement shadows in simulation}{19}{23} \\ % 3 months
\ganttmilestone{Shadows Implementation Milestone}{23} \\
\ganttbar{Finalize and prepare for expo}{24}{27} \\
\ganttlink{elem2}{elem3}
\ganttlink{elem3}{elem4}
\ganttlink{elem4}{elem5}
\ganttlink{elem5}{elem6}
\ganttlink{elem6}{elem7}
\ganttlink{elem7}{elem8}
\end{ganttchart}
\newline
\begin{center}
Gantt Chart for Shady Robots's Senior Design Project
\end{center}
\end{landscape}

\null
\vfill

\noindent\begin{tabular}{ll}
\makebox[2.5in]{\hrulefill} & \makebox[2.5in]{\hrulefill}\\
Cindy Grimm & Date\\[4ex]% adds space between the signatures
\makebox[2.5in]{\hrulefill} & \makebox[2.5in]{\hrulefill}\\
Justin Bibler & Date\\[4ex]% adds space between the signatures
\makebox[2.5in]{\hrulefill} & \makebox[2.5in]{\hrulefill}\\
Matthew Huang & Date\\[4ex]% adds space between the signatures
\makebox[2.5in]{\hrulefill} & \makebox[2.5in]{\hrulefill}\\
Daniel Goh & Date\\
\end{tabular}

\end{document}



