\documentclass[10pt,journal,compsoc]{IEEEtran}

% *** CITATION PACKAGES ***
\ifCLASSOPTIONcompsoc
  % IEEE Computer Society needs nocompress option
  % requires cite.sty v4.0 or later (November 2003)
  \usepackage[nocompress]{cite}
\else
  % normal IEEE
  \usepackage{cite}
\fi

\usepackage{geometry}
\geometry{margin=0.75in}

\begin{document}
\onecolumn
\begin{titlepage}
\title{Better Graphics For A Robotics Grasping GUI\\ Problem Statement \\  [0.5em] 
	\large CS461: Senior Software Engineering Project \\ Fall 2016}

\author{Justin~Bibler,
        Matthew~Huang,
        and~Daniel~Goh}
		
\IEEEtitleabstractindextext{
\begin{abstract}
Our client is conducting a research to improve the robot grasping model.
Data is collected online through the use of visuals demonstrating how a robotic hand grasps objects.
These visuals originate from an existing simulation (OpenRave).
However, the simulation currently has outdated graphics which lessens the reliability of the data collected.
Our plan is to take this existing robot grasping visualization interface and augment it with: warm cool shaders, shadows, and silhouettes.
Warm cool shaders add contrast to objects while still maintaining their visibility.
Shadows highlight the position and movement of objects in the simulation.
Silhouettes make objects visually distinct from one another.
All of these additions will help users understand the geometric relationships between the grasping hand and the object it is trying to pick up. 
\end{abstract}
}

\maketitle

\end{titlepage}

\section{Problem Definition}
Currently, our client is using visualizations, of a robot hand grasping objects, to collect data on-line.
This data is used to create a model of the human grasp.
However, the current graphics in the visualizations are outdated.
It is hard to see and understand the shapes and contact points represented in the scene.
Because of this, users aren't confident in giving proper responses to the simulation.
Thus, the graphics need to be updated to allow meaningful data to be collected.

\section{Proposed Solution}
To improve the robotic grasping visualization in OpenRave, we will be adding in 3 features that help users understand the simulation's geometry and movement. 
These three features are warm cool shaders, shadows and silhouettes. \par

Firstly, we are going to add warm cool shaders that highlight 3D objects in an unrealistic way. 
These shaders are unrealistic because they use unnatural lighting to highlight every part of the object. 
This will allow users to easily understand an object's geometry and depth inside the simulation.
After the shaders are implemented, our focus will be changed to adding shadows to the OpenRave visualization. 
The addition of shadows will make it easier for users to locate the positions of objects and better understand an object's movement in the simulation. 
Finally, we will create silhouettes for the 3D objects to help users better understand distinctions between separate objects in the simulation. \par

We believe that these implementations will allow users to give confident responses to questions about the simulation.
At expo, we will show a demo comparison of the before and after of the robotic grasping simulation. 
Along with this, we will have a poster describing the features we implemented into the simulation. 
Our description will go into detail of how these features were implemented, and why we believe them to be improvements.

\section{Performance Metrics}
Our augmentation will be measured by how aesthetically pleasing the updated visualizations will be.
This will be measured through user interviews.
The users will be composed of faculty members, students and relevant stakeholders.
In the interview, the user will be asked which visual looks better to them.
For the new visuals to be considered a success, at least 80\% of all user answered feedback must choose the new visuals over the old. 
An example of this is ten users are interviewed with ten questions each, the results end with 80 questions answered with the new visuals and 20 with the old; this is considered a success.
There will be multiple questions asked in the interview.
Each question will have two pictures, one picture is the new visual and the other is the old visual.
These questions will be related to how our implementation improves object visibility (shadows, silhouettes, warm/cool), object geometric distinctness (silhouettes, warm/cool), and object understandability in 3d space (shadows, silhouettes).
Furthermore, the simulation should maintain a 30FPS during runtime, and all valid simulation files should be rendered within the first 30 seconds of runtime.
If all criteria has been met, our project will be declared as completed.

\vfill

\noindent\begin{tabular}{ll}
\makebox[2.5in]{\hrulefill} & \makebox[2.5in]{\hrulefill}\\
Cindy Grimm & Date\\[4ex]% adds space between the signatures
\makebox[2.5in]{\hrulefill} & \makebox[2.5in]{\hrulefill}\\
Justin Bibler & Date\\[4ex]% adds space between the signatures
\makebox[2.5in]{\hrulefill} & \makebox[2.5in]{\hrulefill}\\
Matthew Huang & Date\\[4ex]% adds space between the signatures
\makebox[2.5in]{\hrulefill} & \makebox[2.5in]{\hrulefill}\\
Daniel Goh & Date\\
\end{tabular}

\end{document}



