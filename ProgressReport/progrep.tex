\documentclass[10pt,journal,compsoc,draftclsnofoot]{IEEEtran}

% Definition of \subparagraph
\makeatletter
\newcommand\subparagraph{%
  \@startsection{subparagraph}{5}
  {\parindent}
  {3.25ex \@plus 1ex \@minus .2ex}
  {0.75ex plus 0.1ex}
  {\normalfont\normalsize\bfseries}}
\makeatother

\newcounter{subparagraph}[paragraph]

\usepackage{listings}
\usepackage{titlesec}
\usepackage{float}
\usepackage{hyperref}
\usepackage{array}
\usepackage{tocloft}
\usepackage{lscape}
\usepackage{textcomp}
\usepackage{pgfgantt}
\usepackage{amsmath}

\usepackage{geometry}
\geometry{margin=0.75in}

\setcounter{tocdepth}{4}
\setcounter{secnumdepth}{4}

\begin{document}
\onecolumn

\begin{titlepage}
\null
\vspace{15mm}

\begin{flushleft}
\begin{bfseries}
	\vskip2mm
	\Huge{Progress Report for\\ Better Graphics For A Robotics Grasping GUI}\\
	\vspace{15mm}
	\textbf{\huge Shady Robots} \\
	\vskip2mm
	\large{Group 12}
	\vskip5mm
	\Large{Justin Bibler \\
	Matthew Huang \\
	Daniel Goh \\}
\end{bfseries}

\vspace{15mm}
\Large{CS461: Senior Software Engineering Project} \\
\Large{Fall 2016} \\

\vspace{5mm}

\today

\vfill

\begin{normalsize}
{\bf Abstract:}
This document goes over the project our project's purpose and goal, as well as providing an update on our team's progress with the project.
The retrospective section details the positives, deltas, and actions of each week, which will be used from here on out to allow the team to reflect and improve for the remainder of the project timeline.

{\bf Keywords:} OpenInventor, OpenGL, OpenRave, shaders, warm cool shaders, silhouettes, shadows, robotic simulation, geometry, visualization, render, vertex lines, retrospective, positive, delta, action
\end{normalsize}
\end{flushleft}

\newpage

\end{titlepage}

\begin{flushleft}

\section{Project Purpose and Goals}


\section{Where we are in the project}
Our performance metrics for measuring our project's success has been briefly defined in our problem statement document, however they are not fully fleshed out in our requirements document.
We will be revisiting and updating the requirements document as needed.
Also, the lack of available documentations on our project's implementation happened to be a wall that is required for us to break down.
We will need to look further into the source code to fully understand the undocumented implementation.
Besides that, we have a good understanding of the project as a whole and what is required of us throughout the project.

\section{Problems and Solutions}


\section{Interesting Pieces of Code}


\newpage

\section{Retrospective}

\begin{center}
\begin{table}[H]
\caption{Table showing what went right (Positives), what needs to be changed (Deltas), and implementations to fix said Delta (Actions)}
\begin{tabular}{ | p{0.3\linewidth} | p{0.3\linewidth} | p{0.3\linewidth} | }
\hline
\textbf{Positives}  & \textbf{Deltas}  & \textbf{Actions} \\ \hline

Week 2: Met up with our client to introduce ourselves and got an overview and the requirements of the the project. & 
No new deltas were introduced this week. & 
No new deltas. \\ \hline

Week 3: Met our TA, Nels, and got a better understanding of what is expected of us throughout this term. & 
We missed out our team's GitHub wiki post for Week 3. & 
We will be more attentive of the task assigned to us in the future. \\ \hline

Week 4: Revised problem statement based on Kevin's feedback. Learned of the existence of IEEE documents, specifically the IEEE 830-1998 document that is used to write about a project's requirements. & 
No new deltas were introduced this week. & 
No new deltas.  \\ \hline

Week 5: Justin participated in the robot grasping study, we completed a rough draft of the requirements document, and came up with our team name, Shady Robots. & 
The lack of ideas to flesh out the requirements document made it hard for us to write out the document itself.
The draft for the requirements document was done a day before the due date. Because of starting late, we were not able to get input from Kevin, Kirsten or Nels regarding the state of the draft. & 
We will learn better time management skills and start working on documents or action items as soon as possible.
This will give us a breathing space to consult our instructors when we are stuck on a document and get feedback from our instructors to improve the document before submitting it. \\ \hline

Week 6: We further revised problem statement, finished requirements document, Matt an Daniel participated in the robot grasping study, and we received help from Kirsten and Nels for our requirements document
and problem statement & Our requirements document is small, and in our opinion poorly written. & We will be revising the requirements document next term, we will not be taking any requirements out, and we will most
likely add in a few more requirements if we feel that they are necessary.  \\ \hline

Week 7: Final version of problem statement was finished, user interviews were introduced into our project as a new requirement, requirements document was edited further, and received input for Gantt chart and tech review topics. &  
No new deltas were introduced this week. & 
No new deltas. \\ \hline


\end{tabular}
\newline
\label{table:retro}
\end{table}
\end{center}




\end{flushleft}

\end{document}

