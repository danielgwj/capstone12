\documentclass[10pt,journal,compsoc,draftclsnofoot]{IEEEtran}

% Definition of \subparagraph
\makeatletter
\newcommand\subparagraph{%
  \@startsection{subparagraph}{5}
  {\parindent}
  {3.25ex \@plus 1ex \@minus .2ex}
  {0.75ex plus 0.1ex}
  {\normalfont\normalsize\bfseries}}
\makeatother

\newcounter{subparagraph}[paragraph]

\usepackage{listings}
\usepackage{titlesec}
\usepackage{float}
\usepackage{hyperref}
\usepackage{array}
\usepackage{tocloft}
\usepackage{lscape}
\usepackage{textcomp}
\usepackage{pgfgantt}
\usepackage{amsmath}

\usepackage{geometry}
\geometry{margin=0.75in}

\setcounter{tocdepth}{4}
\setcounter{secnumdepth}{4}

\begin{document}
\onecolumn

\begin{titlepage}
\null
\vspace{15mm}

\begin{flushleft}
\begin{bfseries}
	\vskip2mm
	\Huge{Progress Report for\\ Better Graphics For A Robotics Grasping GUI}\\
	\vspace{15mm}
	\textbf{\huge Shady Robots} \\
	\vskip2mm
	\large{Group 12}
	\vskip5mm
	\Large{Justin Bibler \\
	Matthew Huang \\
	Daniel Goh \\}
\end{bfseries}

\vspace{15mm}
\Large{CS461: Senior Software Engineering Project} \\
\Large{Fall 2016} \\

\vspace{5mm}

\today

\vfill

\begin{normalsize}
{\bf Abstract:}
Our customer is using a simulation to create visuals that are used for online data collection.
This simulation is using outdated libraries which result in outdated graphics.


{\bf Keywords:} OpenInventor, OpenGL, OpenRave, shaders, warm cool shaders, silhouettes, shadows, robotic simulation, geometry, visualization, render, vertex lines
\end{normalsize}
\end{flushleft}

\newpage

\end{titlepage}

\begin{flushleft}

\section{Project Purpose and Goals}


\newpage

\section{Where we are in project}


\newpage

\section{Problems and Solutions}




\section{Interesting Pieces of Code}


\newpage

\section{Retrospective}

\begin{center}
\begin{table}[H]
\caption{Table showing what went right (Positives), what needs to be changed (Deltas), and implementations to fix said Delta (Actions)}
\begin{tabular}{ | p{0.3\linewidth} | p{0.3\linewidth} | p{0.3\linewidth} | }
\hline
\textbf{Positives}  & \textbf{Deltas}  & \textbf{Actions} \\ \hline
1 & 1 & 1  \\ \hline
1 & 1 & 1  \\ \hline
1 & 1 & 1  \\ \hline
\end{tabular}
\newline
\label{table:retro}
\end{table}
\end{center}




\end{flushleft}

\end{document}

