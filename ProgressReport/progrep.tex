\documentclass[10pt,journal,compsoc,draftclsnofoot]{IEEEtran}

% Definition of \subparagraph
\makeatletter
\newcommand\subparagraph{%
  \@startsection{subparagraph}{5}
  {\parindent}
  {3.25ex \@plus 1ex \@minus .2ex}
  {0.75ex plus 0.1ex}
  {\normalfont\normalsize\bfseries}}
\makeatother

\newcounter{subparagraph}[paragraph]

\usepackage{listings}
\usepackage{titlesec}
\usepackage{float}
\usepackage{hyperref}
\usepackage{array}
\usepackage{tocloft}
\usepackage{lscape}
\usepackage{textcomp}
\usepackage{pgfgantt}
\usepackage{amsmath}

\usepackage{geometry}
\geometry{margin=0.75in}

\setcounter{tocdepth}{4}
\setcounter{secnumdepth}{4}

\begin{document}
\onecolumn

\begin{titlepage}
\null
\vspace{15mm}

\begin{flushleft}
\begin{bfseries}
	\vskip2mm
	\Huge{Progress Report for\\ Better Graphics For A Robotics Grasping GUI}\\
	\vspace{15mm}
	\textbf{\huge Shady Robots} \\
	\vskip2mm
	\large{Group 12}
	\vskip5mm
	\Large{Justin Bibler \\
	Matthew Huang \\
	Daniel Goh \\}
\end{bfseries}

\vspace{15mm}
\Large{CS461: Senior Software Engineering Project} \\
\Large{Fall 2016} \\

\vspace{5mm}

\today

\vfill

\begin{normalsize}
{\bf Abstract:}
This document goes over the project our project's purpose and goal, as well as providing an update on our team's progress with the project.
The retrospective section details the positives, deltas, and actions of each week, which will be used from here on out to allow the team to reflect and improve for the remainder of the project timeline.

{\bf Keywords:} OpenInventor, OpenGL, OpenRave, shaders, warm cool shaders, silhouettes, shadows, robotic simulation, geometry, visualization, render, vertex lines, retrospective, positive, delta, action
\end{normalsize}
\end{flushleft}

\newpage

\end{titlepage}

\begin{flushleft}

\section{Project Purpose and Goals}
The purpose of our project is to update the graphics of an existing program that creates a simulation of a robotic arm grasping objects.
The reasoning behind our desire to update the graphics is because our client is using online data collection methods using these visualizations.
However, the outdated graphics are making it difficult for users to give confident responses to the survey.
To fix this problem, we have created 4 major goals:
\begin{enumerate}
\item Implement Gooch shading.
\item Implement shadows.
\item Implement silhouettes.
\item Test to ensure our implementations improve user confidence using interviews.
\end{enumerate}

\section{Where we are in the project}
In our current state, the hard requirements for our project and the designs for how we'll be implementing them are clear to us.
Our design document has a plan of action for every requirement our client needs.
Our requirements document, however, needs to be updated.
Currently, it does not include any stretch goals that we may have and its introduction needs more context.
Additionally, certain requirements (such as Gooch Shading) needs more detail added to it.
Past that, we also need to add our updated performance metrics on top of our current requirements.
This document was been put on hold due to the design document, but it will be updated as soon as we have time.
As far as coding goes, we have not made any modifications to the current source code; we have simply looked through it.
Our initial goal was to ascertain how the parts in the system (OpenRave, OpenInventor, Qt) interacted with each other.
Our current understanding is that OpenInventor contains object libraries which OpenRave pulls from, OpenRave creates the simulation environment and renders the 3D objects, and Qt simply creates the window that displays on the monitor.
It is important to note that our design document is written based on this understanding.
We've also attempted to get OpenRave running on our own machines; Justin in particular has devoted several hours to doing this.
Unfortunately, however, we have yet to get the current system running due to dependency issues.

\section{Problems and Solutions}
Throughout the term, we have encountered many problems as we compose the requirements document, technology review document, and design document.
When writing the requirements document, we were unsure how we could incorporate a project as small as ours and turn it into a full fledged document that follows the IEEE Std 830-1998.
With the lack of consultation, we ended up writing a poor requirements document.
To prevent this from happening again, our solution was to assign our individual parts ahead of time, which gives us time to consult our instructors if necessary.
This was implemented during the technology review assignment.
We split up our tasks early on, and was able to get feedback from Nels regarding the structure and the expectations for the design document.
This placed us in a better position as we are more confident and aware of what was expected of us. \\
\vspace{3mm}
The other problem that we encountered for this project was regarding the outdated OpenGL libraries, which capabilities to implement modern looking graphics.
As the existing system utilizes OpenGL 1.0 libraries, we were unsure if we would need to rewrite the system as a whole or if there is an easy way to upgrade the current library to a more modern version of the OpenGL library.
Our member took the initiative and visited Professor Bailey in his office to inquire about our constraint.
After consulting Professor Bailey, it appears that upgrading the OpenGL libraries would not be a hard endeavor on our side.
Knowing that, our team was more relieved of the library upgrading process. \\
\vspace{3mm}
The project problem that we are facing even now, is that one of the subsystem that is used for the simulation is not open sourced.
This is a problem as our client suspects that the main render loop for the visualization is within the subsystem that is not open sourced.
This would require use to dig into the code base ourselves and figure out a way to do some reverse engineering.
The solution to this problem is not clear yet, but a preliminary plan is formed.
This plan is to familiarize ourselves with the existing source code and to identify the render loop.
This is reflected in the Gantt Chart that we have drawn for the requirements document. \\

\newpage

\section{Retrospective}

\begin{center}
\begin{table}[H]
\caption{Table showing what went right (Positives), what needs to be changed (Deltas), and implementations to fix said Delta (Actions)}
\begin{tabular}{ | p{0.3\linewidth} | p{0.3\linewidth} | p{0.3\linewidth} | }
\hline
\textbf{Positives}  & \textbf{Deltas}  & \textbf{Actions} \\ \hline

Week 2: Met up with our client to introduce ourselves and got an overview and the requirements of the the project. & 
No new deltas were introduced this week. & 
No new deltas. \\ \hline

Week 3: Met our TA, Nels, and got a better understanding of what is expected of us throughout this term. & 
We missed out our team's GitHub wiki post for Week 3. & 
We will be more attentive of the task assigned to us in the future. \\ \hline

Week 4: Revised problem statement based on Kevin's feedback. Learned of the existence of IEEE documents, specifically the IEEE 830-1998 document that is used to write about a project's requirements. & 
No new deltas were introduced this week. & 
No new deltas.  \\ \hline

Week 5: Justin participated in the robot grasping study, we completed a rough draft of the requirements document, and came up with our team name, Shady Robots. & 
The lack of ideas to flesh out the requirements document made it hard for us to write out the document itself.
The draft for the requirements document was done a day before the due date. Because of starting late, we were not able to get input from Kevin, Kirsten or Nels regarding the state of the draft. & 
We will learn better time management skills and start working on documents or action items as soon as possible.
This will give us a breathing space to consult our instructors when we are stuck on a document and get feedback from our instructors to improve the document before submitting it. \\ \hline

Week 6: We further revised problem statement, finished requirements document, Matt an Daniel participated in the robot grasping study, and we received help from Kirsten and Nels for our requirements document
and problem statement & Our requirements document is small, and in our opinion poorly written. & We will be revising the requirements document next term, we will not be taking any requirements out, and we will most
likely add in a few more requirements if we feel that they are necessary.  \\ \hline

Week 7: Final version of problem statement was finished, user interviews were introduced into our project as a new requirement, requirements document was edited further, and received input for Gantt chart and tech review topics. &  
No new deltas were introduced this week. & 
No new deltas. \\ \hline

Week 8: Completed tech review and made minor edits to requirements document. & 
We should make sure everyone is clear on the format of the paper before we start writing. &
Have a clear outline up for how we'll write the document before any of us start writing. \\ \hline

Week 9: Briefly looked over IEEE Std 1016-2009 document. &
We should rarely have a week where almost nothing is done unless under extreme circumstances. &
Make sure we do something worth writing down every week unless circumstances are extraneous. \\ \hline

Week 10: Completed Design Document. Started working on written and presentation portions of progress report. &
No new deltas were introduced this week. &
We should make sure everyone understands the outlines we put up on github. \\ \hline

\end{tabular}
\newline
\label{table:retro}
\end{table}
\end{center}




\end{flushleft}

\end{document}

