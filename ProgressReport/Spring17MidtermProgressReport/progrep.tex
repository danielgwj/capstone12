\documentclass[10pt,journal,compsoc,draftclsnofoot]{IEEEtran}

% Definition of \subparagraph
\makeatletter
\newcommand\subparagraph{%
  \@startsection{subparagraph}{5}
  {\parindent}
  {3.25ex \@plus 1ex \@minus .2ex}
  {0.75ex plus 0.1ex}
  {\normalfont\normalsize\bfseries}}
\makeatother

\newcounter{subparagraph}[paragraph]

\usepackage{listings}
\usepackage{titlesec}
\usepackage{float}
\usepackage{hyperref}
\usepackage{array}
\usepackage{tocloft}
\usepackage{lscape}
\usepackage{textcomp}
\usepackage{pgfgantt}
\usepackage{amsmath}

\usepackage{geometry}
\geometry{margin=0.75in}

\setcounter{tocdepth}{4}
\setcounter{secnumdepth}{4}

\begin{document}
\onecolumn

\begin{titlepage}
\null
\vspace{15mm}

\begin{flushleft}
\begin{bfseries}
	\vskip2mm
	\Huge{Midterm Progress Report for\\ Better Graphics For A Robotics Grasping GUI}\\
	\vspace{15mm}
	\textbf{\huge Shady Robots} \\
	\vskip2mm
	\large{Group 12}
	\vskip5mm
	\Large{Justin Bibler \\
	Matthew Huang \\
	Daniel Goh \\}
\end{bfseries}

\vspace{15mm}
\Large{CS463: Senior Software Engineering Project} \\
\Large{Spring 2017} \\

\vspace{5mm}

\today

\vfill

\begin{normalsize}
{\bf Abstract:}
This document goes over the project our project's purpose and goal, as well as providing an update on our team's progress with the project.

{\bf Keywords:} OpenRAVE, shaders, warm cool shaders, silhouettes, shadows, robotic simulation, geometry, visualization, render, vertex lines
\end{normalsize}
\end{flushleft}

\newpage

\end{titlepage}

\begin{flushleft}

\section{Project Purpose and Goals}
The purpose of our project is to update the graphics of an existing program that creates a simulation of a robotic arm grasping objects.
The reasoning behind our desire to update the graphics is because our client is using online data collection methods using these visualizations.
However, the outdated graphics are making it difficult for users to give confident responses to the survey.
To fix this problem, we have created 4 major goals:
\begin{enumerate}
\item Implement Gooch shading.
\item Implement shadows.
\item Implement silhouettes.
\item Test to ensure our implementations improve user confidence using interviews.
\end{enumerate}

\section{Where we are in the project}



\section{What is left to do}

\newpage

\section{Problems and Solutions}
\subsection{Justin Bibler}
\textbf{Problem1}
\par
placeholder placeholder placeholder placeholder placeholder placeholder
placeholder placeholder placeholder placeholder placeholder placeholder
\par
placeholder placeholder placeholder placeholder placeholder placeholder
placeholder placeholder placeholder placeholder placeholder placeholder
\par
placeholder placeholder placeholder placeholder placeholder placeholder

\newpage

\subsection{Matthew Huang}
\textbf{Problem1}
\par
By the end of our beta release, we had implemented warm cool shaders, silhouettes, and shadows together.
As a reminder, this required 2 render passes since the silhouettes need to be rendered separate from the other shaders.
Additionally, due to API limitations, we had to inject the warm cool shader code into the SoShadowGroup library so that the shadows would not override the augmented warm cool colors.
At this point, we were feature complete and all that was left to do was polishing based on client feedback.
When we demoed the project to Cindy, she suggested that we make alterations to each feature so that they were all more consistent across examples.
Specifically, she wanted the warm cool colors to adapt based on object color, more consistent silhouette widths across different shapes, and darker shadows.
Since I am responsible for the warm cool shaders and silhouettes, I will outline how those features were repolished; I will leave the darker shadows for Justin to explain.
Do note, however, that Justin and I largely colaborated in implementing all 3 feature changes.
\vspace{3mm}

\textbf{Adapative Warm Cool Shaders}
To implement the warm cool color adaptation, we set color ranges that would catch each different object color we wanted to account for.
The colors we ended up accounting for were red, green, blue, yellow, purple, cyan, white, and grey/black.
After we caught one of the colors, we would then set specific warm cool colors that would blend well with object color.
For example, the warm cool colors for red are yellow and red respectively while the warm cool colors for green are green and blue respectively.
This allowed the scene to retain much of its original colors while also having the warm cool color ranges.
Additionally, this prevented bad color combinations such as blue and orange.

\vspace{3mm}

\textbf{More consistent Silhouettes}

For the more consistent silhouettes, we found that the solution was to render the silhouette using polygonal lines (GL\_LINES) instead of solid polygons.
By doing this, every shape in the scene would have its own silhouette who's width was based on the set line width.
Then, changing the line width was simple; all we had to do was increase the line width value within our draw style module.
Additionally, to fix the silhouette overlapping issue, we also added a polygonal offset module to our silhouette render. 
This prevented the silhouettes from being at the same depth as the warm cool shaded model so any overlapping issue was largely resolved.
Also note that front and back face culling for the warm cool render and the silhouettes respectively were still being done.
Finally, the final silhouette color we choose was black.
This was our choice, despite the fact that they blended into the background, because silhouette are meant to simply help differentiate objects from each other.


\textbf{Testing Performance}
\par
\vspace{3mm}
While implementing silhouettes, we ran into a line width runtime error that was causing our line widths to be immutable meaning our line widths would be stuck at 1.
What was interesting, though, was that this error only occurred while running the simulation on Justin and Daniel's machines, not mine.
What we found was that our virtual machines were accessing our laptop's hardware differently.
Namely, there's was using 3D hardware acceleration while mine was not.
Turning this option off fixed the line width problem, but hurt our program's performance.
Specifically, it dropped our frames per second from around 30-40 fps to 10 fps.
Our intuition told us that this was caused by our virtual machines being unable to properly use all of the hardware on our laptops.
To fix this, we had to install Ubuntu 16.04 natively on one of our laptops.
We switched to 16.04 from 14.04 because 14.04 had driver issues that prevented us from being able to access the internet.
This fixed our fps issues (now sitting at around 40 - 50 fps average) while still retaining the mutable line widths for our silhouettes.
\newpage

\subsection{Daniel Goh}
\textbf{Problem1}
\par
placeholder placeholder placeholder placeholder placeholder placeholder
\par
placeholder placeholder placeholder placeholder placeholder placeholder

\vspace{3mm}

\textbf{Problem2}
\par
placeholder placeholder placeholder placeholder placeholder placeholder
\par
placeholder placeholder placeholder placeholder placeholder placeholder

\newpage

\null
\vfill

\end{flushleft}

\end{document}

