\documentclass[10pt,journal,compsoc]{IEEEtran}

% *** CITATION PACKAGES ***
\ifCLASSOPTIONcompsoc
  % IEEE Computer Society needs nocompress option
  % requires cite.sty v4.0 or later (November 2003)
  \usepackage[nocompress]{cite}
\else
  % normal IEEE
  \usepackage{cite}
\fi

\usepackage{geometry}
\geometry{margin=0.75in}

\begin{document}
\onecolumn
\begin{titlepage}
\title{Better Graphics For A Robotics Grasping GUI\\ Problem Statement \\  [0.5em] 
	\large CS461: Senior Software Engineering Project \\ Fall 2016}

\author{Justin~Bibler,
        Matthew~Huang,
        and~Daniel~Goh,}
		
\IEEEtitleabstractindextext{%
\begin{abstract}
The current OpenRave simulation has outdated graphics. 
The goal is to take this existing robot grasping visualization interface and augment it with warm cool shaders, shadows and silhouettes. 
This will help users understand the geometric relationship between the grasping hand and the object it is trying to pick up. 
We’ll be doing this by incorporating shaders, shadows, and silhouettes into the existing simulation.
\end{abstract}
}

% make the title area
\maketitle

\end{titlepage}

\section{Problem Definition}
The robot grasping model that is within OpenRave does not meet the client’s needs. 
It is currently difficult to see and understand the shapes and contact points within the simulation due to outdated graphics. 
Because of this, users are unwilling or unable to give proper responses to the simulation.

\section{Proposed Solution}
To improve the robotic grasping visualization in OpenRave, 
we will be adding in 3 features that help users understand the simulation’s 
geometry and movement. These 3 features are shaders, shadows and silhouettes. \par
Firstly, we are going to add warm cool shaders that highlight 3D objects in an 
unrealistic way so that users can easily understand the shape of an object. 
After the shaders are implemented, our focus will be changed to adding shadows 
to the OpenRave visualization. The addition of shadows will make it easier for 
users to locate the positions of objects and better understand an object’s movement 
in the simulation. Finally, we will create silhouettes for the 3D objects to help 
users better understand it’s geometry within the simulation. \par
We will show a demo comparison of the before and after at expo in addition to a 
poster that detail what we’ve added to improve the visualization of the simulation.

\section{Performance Metrics}
We will be using the 3 features that we have proposed as our performance metrics. 
It will be based on whether or not the 3 features are implemented within the 
simulation. We will be using our client’s response to determine if the 3 
features were implemented correctly.

\end{document}


